%&pdflatex
\documentclass[a4paper]{article}

%% Language and font encodings
\usepackage[english]{babel}
\usepackage[utf8x]{inputenc}
\usepackage[T1]{fontenc}

%% Sets page size and margins
\usepackage[a4paper,top=3cm,bottom=2cm,left=3cm,right=3cm,marginparwidth=1.75cm]{geometry}

%% Useful packages
\usepackage{amsmath}
\usepackage{graphicx}
\usepackage[colorinlistoftodos]{todonotes}
\usepackage[colorlinks=true, allcolors=blue]{hyperref}

\title{Your Paper}
\author{Amelia Gordafarid Crowther}

\begin{document}

\begin{abstract}
Your abstract.
\end{abstract}

\section{Introduction}

\section{Background}

\subsection{Kinect Camera}

talk about input to system, explain rgbd etc

single camera used

commodity level, not specialist equipment, available directly to consumers

cheap and easily available

depth information good for price

30Hz 640x480 colour and depth

errors and missing depth information for pixels a problem

\subsection{Static Reconstruction}

\subsubsection{KinectFusion}

truncated signed distance field

iterative closest point (ICP) algorithm takes two point clouds and returns the rigid transformation which aligns them best


\subsubsection{static example 2}

\subsubsection{static example 3}

\subsection{Offline Non-rigid Reconstruction}


talk about immense demands here, how its actually really hard to get it all done in real time

\subsubsection{People doing the group project}

implementation of dynamic fusion but offline

\subsubsection{completion of dynamic shapes}

no canonical model, only compares with models in adjacent frames


\subsubsection{offline example 3}

\subsection{Real-time Non-rigid Reconstruction}

several methods have been proposed for performing reconstruction of a non-rigid surface in real-time.
here are a few:

\subsubsection{Dynamic Fusion}
based off kinectfusion

\subsubsection{Killing Fusion}

killing vector field

\subsubsection{Volume Deform}


\section{Technical Details}

\section{Evaluation}

\section{Conclusions}

\section{Appendix}

\bibliographystyle{plain}
\bibliography{report}

\end{document}
